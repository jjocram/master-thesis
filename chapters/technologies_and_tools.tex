\documentclass[../thesis.tex]{subfiles}
\begin{document}
\chapter{Technlogies and tools}\label{cap:technologies-and-tools}

\section{MediaPipe}
MediaPipe is an open-source, real time and on-device tool that can track multiple part of the body. In particular I am interested in hands tracking. MediaPipe suits very well for this purpose because it offers a pre-trained \acrshort{ML} model to recongize and track twenty one landmarks on each hand. In particular it uses a pipeline composed by two \acrshort{ML} models:
\begin{enumerate}
    \item a palm detector that works on a full image, locates the palm and identifies the bounding box around it;
    \item a hand landmark model that works on the cropped image of the palm and returns the hand landmarks considering the depth also. 
\end{enumerate}
The precision of this tool is about the $96\%$~\cite{article:mediapipe} so, it isa good starting point for the hand gesture recognition task. It is possible get the posisition of the landmarks and give them in input at a Deep Neural Network trained on the gestures of our interest.\\

It is interesting to point out that MediaPipe is capable to track, in real time, different parts of a human body, for example the face and the whole body.~\cite{site:mediapipe}

\section{Tensorflow}
Tensorflow is an open source Python library to build \acrshort{ML} models. Google started its developing in 2015 and today is one of the most used Python libraries to perform \acrshort{ML} tasks. It offers a huge amount of layers, activation functions and tools to build simple and complex neural network architectures.

\subsection{Tensorflow Lite}

\section{Robot Operating System}
 


\end{document}

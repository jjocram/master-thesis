    \documentclass[../thesis.tex]{subfiles}
\begin{document}
\chapter{Background}\label{cap:background}

\section{Human-Robot Interaction}

\section{Hand gesture recognition}
The hand gesture recognition task is a well-studied task. In the literature is possible to find the idea to use the hand gesture as a way to interact with a machine since 1987~\cite{article:hand_gesture_interface_device}. At that time the idea was to use a glove to recognize the position and the orientation of the user's hand. They were thought to be used for different tasks like gesture recognition, an interface to a visual programming language, virtual object manipulation, and many others. Nowadays, the idea to use a third-party device to recognize what the hands are doing is losing interest in the community, especially thanks to the increase of computing power and the improved quality of video acquisition devices. Thanks to these improvements, and the study of computer vision and machine learning techniques, it is possible to reach a good level of accuracy with only a webcam.
% Talk about datasets, evaluation methods and what are the latest results

\end{document}
